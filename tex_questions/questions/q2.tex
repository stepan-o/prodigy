\chapter{Incorporate expert rating of difficulty of each task} \label{ch:q2_expert_rating}

\section{Question 2} \label{sec:q2}

Now suppose you are given a set $L$ of task lists $L_1, L_2, \dots, L_n$.
Each task list contains some tasks in estimated ascending difficulty order as measured by some integer $D$, where $D = 0$ for an extremely trivial task and $D = 50$ for the hardest task imaginable.
However, these given task difficulties are only estimates by experts - in practice each user who tries a task might find that task easier or harder than the expert thought it would be.
How would you use the given set of task lists along with the actual performance of users on each task to generate a single combined task list with better estimated difficulties for all tasks?
(Please assume that for any task $T_i$, the difficulty estimate $D_i$ can vary depending on the specific expert's evaluation of difficulty for $T_i$ - that is to say, experts will likely disagree on how difficult any given task is.
You can also assume that the same task may appear on multiple task lists.)

\section{Solution} \label{sec:q2_solution}

\subsection{Combining individual expert assessments into a single expert difficulty estimate $D_{expt}(T_k)$} \label{subsec:q2_dexp}

First, it is necessary to obtain the combined expert difficulty estimates $D_{expt}(T_k)$ for each task $T_k$ in $T$ by combining all the ratings $D_i(T_k)$ found for a given task on different task lists in $L$.
Since no specific weight is given to individual expert opinions, the combined expert difficulty estimate $D_{expt}(T_k)$ can be taken for each task $T_k$ as the mean of all estimates $D_i(T_k)$ from every task list in $L$ which contains the task $T_k$:

\begin{equation} \label{eq:d_expt}
    D_{expt}(T_k) = \left \lfloor \frac{1} {m} \sum \limits_{i=1}^m D_i(T_k) \right \rfloor
\end{equation}

where $m$ is the count of task lists in $L$ which include the task $T_k$.

In the equation~\ref{eq:d_expt}, $D_{expt}(T_k)$ can be left as a real number and be directly converted to a probability of a correct answer using the equation~\ref{eq:dexpt_to_dabs_conversion} that will be introduced in the following section.
However, the preferable way of converting the 0--50 scale used by experts to mark $D_{expt}$ to a 0--1 scale comparable with probabilities expressed by $D_{abs}$ involves grouping questions by $D_{expt}$(to be discussed in the following section), and thus works better for a discrete set of values.
For this purpose in equation~\ref{eq:d_expt}, the mean of all expert estimates is rounded down to the nearest integer.
This process introduces a pessimistic bias by rounding the mean difficulty values down, but since the task difficulty scale used by experts ranges from 0 to 50, this effect is assumed to be mild.

\subsection{Combining empirically determined absolute task difficulty $D_{abs}(T_k)$ with expert estimates $D_{expt}(T_k)$} \label{subsec:q2_dcomp}

In order to obtain the composite absolute difficulty measure $D_{comp}(T_k)$ of a task $T_k$, the two independently obtained measures of task difficulty $D_{abs}(T_k)$ and $D_{expt}(T_k)$ need to be combined into a single measure;
to facilitate this, they need to be converted to a comparable scale.
As was discussed in section~\ref{subsec:q1_difficulty_index}, $D_{abs}(T_k)$ represents the empirically obtained expected value of the probability that an average user in $U$ can solve the task $T_k$ (as observed by performance of users in the subset $U_{T_k}$).
In contrast, expert task difficulty estimate $D(T_k)$ is assigned using an integer scale from 0 to 50, with 0 corresponding to an extremely trivial task and 50 representing the hardest task imaginable.
Since $D_{expt}(T_k)$ was defined in section~\ref{subsec:q2_dexp} as the rounded down mean of all $D_i$ for a given task $T_k$, it will also fall onto an integer scale from 0 to 50.

To bring the two metrics onto a single composite absolute difficulty measure, $D_{comp}(T_k)$, the scale of $D_{expt}$ must be mapped onto the corresponding probabilities $D_{expt\_prob}$ of the correct answer by an average user in $U$, similar to the ones that were estimated empirically for $D_{abs}(T_k)$ in section~\ref{subsec:q1_difficulty_index}:

\begin{equation}
    \begin{split}
        D_{expt\_prob}(T_k | D_{expt}(T_k) = D_i) = E[P(Ans=Corr | D_{expt}(T_k) = D_i)]
    \end{split}
\end{equation}

The most simple way of converting the 0--50 difficulty scale used by experts to a 0--1 probability scale is through linear interpolation.
$D(T_k)=0$ can be assumed to correspond to difficulty $D_{abs}(T_k)=1.00$ (all users can solve the most trivial task), and $D(T_k)=50$ to correspond to $D_{abs}(T_k)=0.0$ (no user can solve the hardest task imaginable).
This linear relationship between expert difficulty estimate $D_{expt}$ and the absolute probability of the correct answer $D_{abs}=p$ can be represented as follows:

\begin{equation} \label{eq:dexpt_to_dabs_conversion}
    D_{expt\_prob}(T_k | D_{expt}(T_k) = D_i) = 1 - 0.02 \cdot D_{expt}
\end{equation}

where $D_i$ is an integer value from 0 to 50.

However, a better approach to converting expert difficulty estimates $D_{expt}(T_k)$ into probabilities $D_{expt\_prob}(T_k)$ is to calibrate expert task difficulty estimates to actual performance of users in $U$.
Doing this empirically would require mapping each discrete value of $D$ to the corresponding ratio of correct answers to all attempts for all the tasks in $T$ with the difficulty level of $D_i$;
this will allow to capture the non-linear relationship between expert difficulty estimates and performance of users in $U$.
This process could be repeated in order to provide an updated calibration of static expert difficulty estimates to evolving user population.

After both task difficulty measures $D_{abs}(T_k)$ and $D_{expt}(T_k)$ have been brought to the same scale (expected probabilities of the correct answer given by a user from $U$), the final composite task difficulty measure can be defined as the mean of the two difficulty measures:

\begin{equation}
    D_{comp}(T_k) = \frac{D_{abs}(T_k) + D_{expt\_prob}(T_k | D_{expt}(T_k) = D_i)} {2}
\end{equation}

Alternatively, if a given preference exists between the two difficulty measures, $D_{comp}$ can be defined as the weighted sum of the two.
For example, the uncertainty of the $D_{abs}(T_k)$ arising from the limited size of the sample of users $U_{T_k}$ who attempted the task $T_k$ (discussed in section~\ref{subsec:q1_dabs_uncertainty}) can be used as a weight coefficient;
while the number of users who tried solving $T_k$ remains small, more weight can be given to expert difficulty estimate;
as the number of users in $U_{T_k}$ grows, the role of expert assessment diminishes with more reliance being put on empirical data.

\subsection{Converting absolute composite task difficulty $D_{comp}(T_k)$ into a discrete random variable $D_{cat}(T_k)$} \label{subsec:q2_discrete_conversion}

The main task of the adaptive algorithm for designing custom task lists for each user session (to be discussed in the following sections) is to determine the sequence of task difficulties to be presented to a user.
Since for the purposes of the implementation of a planning agent it is more convenient to work with discrete action spaces rather than with continuous ones, question difficulty can be categorized by binning the values of $D_{comp}$ into a discrete categorical absolute difficulty $D_{cat}$ using a rough ``rule-of-thumb'' classification according to the difficulty index, as is shown in table~\ref{tab:difficulty_index}.

\begin{table}[h!]
    \centering
    \begin{tabular}{ c | c | c }
        \toprule
        \hline
        Task difficulty ($D_{cat}(T_k)$) & $D_{comp}(T_k)$ & \% correct \\
        \midrule
        \hline
        Very trivial & $p > 0.8$ & Over 80\% \\
        \hline
        Trivial & $0.6 < p \leq 0.8$ & From 60\% to 80\% \\
        \hline
        Moderate & $0.4 < p \leq 0.6$ & From 40\% to 60\% \\
        \hline
        Hard & $0.2 < p < \leq 0.4$ & From 20\% to 40\% \\
        \hline
        Very hard & $p \leq 0.2$ & Less than 20\% \\
        \hline
        \bottomrule
    \end{tabular}
    \caption{Categorizing questions by absolute difficulty using difficulty index.}
    \label{tab:difficulty_index}
\end{table}

So far, all the consideration has been given to rating absolute task difficulty, as defined by empirical user performance or expert judgement.
However as was discussed in section~\ref{subsec:q1_abs_rel_difficulty}, due to the difference in ability levels shown by users in $U$ on the tasks in $T$, there are two definitions of task difficulty that need to be considered for the purposes of custom task list design: absolute task difficulty for a given task $T_k$ and relative task difficulty for a given pair of task $T_k$ and user $U_i$.
In the following section, the absolute categorical task difficulty $D_{cat}(T_k)$ that was produced from the composite absolute task difficulty index $D_{comp}(T_k)$ will be used in combination with the user ranking by experience to produce relative task difficulty, $D_{rel}(T_k, U_i)$.

\subsection{Pseudocode} \label{subsec:q2_pseudocode}

\# combine expert estimates into a single absolute difficulty assessment

for each task $T_k$ in $T$:

\vspace{5mm}

$\quad$ Initialize empty list of ratings

$\quad$ for each task list in $L$ containing $T_k$:

$\quad\quad$ append rating $D_i(T_k)$ to the list of ratings

$\quad$ $D_{expt}(T_k) = \left \lfloor \frac{1} {m} \sum \limits_{i=1}^m D_i(T_k) \right \rfloor$, where $m$ in the number of task lists in $L$ containing task $T_k$

\vspace{5mm}

$\quad$ \# convert from $D_{expt}$ 0--50 scale to $D_{expt\_prob}$ 0--1 scale using linear relationship

$\quad$ \# Note: preferred conversion method involves calibration of $D_{expt\_prob}$ estimates using empirical data

\vspace{5mm}

$\quad$ $ D_{expt\_prob}(T_k | D_{expt}(T_k) = D_i) = 1 - 0.02 \cdot D_{expt} $

\vspace{5mm}

$\quad$ \# determine the combined absolute task difficulty

\vspace{5mm}

$\quad$ $D_{comp}(T_k) = \frac{D_{abs}(T_k) + D_{expt\_prob}(T_k | D_{expt}(T_k) = D_i)} {2}$
