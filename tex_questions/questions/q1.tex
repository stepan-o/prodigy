\chapter{Estimate the difficulty of each task} \label{ch:q1_difficulty}

\section{Question 1} \label{sec:q1}

You want to estimate the difficulty of each task.
Each user is given a random sample of tasks from $T$ to attempt.
How can you use the performance of users (on the random task sample they are given) to estimate the difficulty of all the tasks in $T$?
(You can only use observations on the computational tasks - no biometrics, direct observation, etc.
Imagine you can see what is on a user's computer screen but nothing else.)

\section{Solution} \label{sec:q1_solution}

\subsection{Absolute and relative difficulty of a task $T_k$} \label{subsec:q1_abs_rel_difficulty}

Since it is said that users in $U$ have different abilities on tasks in $T$, when discussing the assessment of the difficulty of a task $T_k$ there are two distinct definitions of difficulty that have to be considered:

\begin{enumerate}
    \item $D_{abs}(T_k)$: absolute measure of difficulty for task $T_k$

    One of the ways to define $D_{abs}(T_k)$ is as $p(T_k)$, or empirically observed percentage of correct responses from all the answers given by the subset of users $U_{T_k}$ who attempted the given task $T_k$.
    This measure indicates the difficulty of a given task as expressed by the ratio of users who were able to solve it correctly.

    \item $D_{rel}(T_k)$: relative measure of difficulty for task $T_k$

    $D_{rel}(T_k)$ can be defined as $D_{abs}(T_k)$ corrected for the current level of abilities of a given user.
    This measure reflects the fact that the two users $U_i$ and $U_j$ might find the same task $T_k$ to have different level of difficulty, depending on their current abilities in a given subject.
    Furthermore, the same user $U_i$ can find the same task $T_k$ to have higher or lower level of difficulty at different points in time (e.g., if the user $U_i$ improves their skills with the subject through practice).
    $D_{rel}(T_k)$ is intended to reflect both differences when assessing the relative difficulty of a given task $T_k$.
\end{enumerate}

This section is concerned with the absolute difficulty, $D_{abs}(T_k)$, of a task as reflected by the performance of the subset of users $U_{T_k}$ who have attempted the task $T_k$.
Relative task difficulty $D_{rel}(T_k)$ will be discussed further in the following sections when introducing the question-feeding bot.

\subsection{Determining absolute task difficulty $D_{abs}(T_k)$ using difficulty index} \label{subsec:q1_difficulty_index}

Difficulty index presents a basic approach to determining absolute task difficulty: difficulty $p$ of each task $T_k$ from a given school year is determined as the fraction of the correct responses provided by all the users who attempted it.
Difficulty $p$ of a task $T_k$ can be determined as the following ratio:

\begin{equation}
    \label{eq:task_difficulty}
    D_{abs}(T_k) = p(T_k) = \frac{\text{Correct answers}} {\text{Total answers}}
\end{equation}

This approach assumes that each task in $T$ was only shown to the subset of users $U_{T_k}$, for whom this task is relevant (e.g., tasks and users are grouped by school year).
Also, in this approach, individual skill level of users is not taken into account;
instead, absolute task difficulty $D_{abs}(T_k)$ can be interpreted as the expectation of the probability of an average user $U_i$ solving the given task $T_k$ correctly, as reflected by the subset of users $U_{T_k}$ who attempted this task:

\begin{equation}
    D_{abs}(T_k) = E[P(Ans=Corr)|U_i \in U_{T_k}]
\end{equation}

\subsection{Uncertainty due to the size of subsets $U_{T_k}$} \label{subsec:q1_dabs_uncertainty}

Since in our case each task $T_k$ is only attempted by a subset of users $U_{T_k}$, the assumption of this definition of absolute task difficulty $D_{abs}(T_k)$ is that the expectation of probability of correct answer for an average user from the sample $U_{T_k}$ is close to the expectation of probability of correct answer for an average user from the whole population $U$.

\begin{equation} \label{eq:dabs_assumption}
    E[P(Ans=Corr|U_i \in U_{T_k})] \approx E[P(Ans=Corr|U_i \in U)]
\end{equation}


Since it is said that users are given randomized subsets of tasks from $T$, given the large number of users and tasks, it can be assumed that for most tasks in $T$, each task $T_k$ is shown to a random sample of users $U_{T_k}$ whose skill level and platform usage patterns are representative of the whole population of users in $U$;
thus, absolute difficulty $D_{abs}(T_k)$ of the task $T_k$ can be assumed to offer an unbiased estimate of task difficulty with respect to all users in $U$.

However, this assumption is more likely to hold true in cases where the subset of users $U_{T_k}$ is sufficiently large, that is, enough users have attempted to solve the task $T_k$.
%TODO determine the uncertainty of this method (size of the user sample U_T_k compared to the size of the whole user population U)
To determine the uncertainty of this method, distribution of sample sizes $U_{T_k}$ can be compared with the size of the population $U$ for all tasks in $T$.
In addition, analysis of the distribution of $D_{abs}$ obtained empirically could be used to test this assumption and inform the design of a more meaningful scoring system that can be used to determine the absolute difficulty of a task.

%TODO add psudocode