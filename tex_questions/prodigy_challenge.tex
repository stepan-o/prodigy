%! Author = Stepan Oskin
%! Date = 14-Dec-19

% Preamble
\documentclass[11pt]{report}

% Packages
\usepackage{amsmath, subcaption}
\usepackage{graphicx}
\usepackage{titlesec, blindtext, color}
\usepackage{xcolor}

% New commands
\newcommand{\norm}[1]{\left\lVert#1\right\rVert}
\newcommand{\vect}[1]{\boldsymbol{#1}}
\newcommand{\pa}[1]{\partial{#1}}

\titleformat{\chapter}[hang]{\Huge\bfseries}{Question \thechapter\hsp\textcolor{gray75}{|}\hsp}{0pt}{\Huge\bfseries}

% Document
\begin{document}

    \title{Prodigy Algorithmic Design Challenge}

    \author{Stepan Oskin}

    \maketitle

    \begin{abstract}

        You are given a list of tasks $T = T_1, T_2, \dots, T_n$.
        Each task is to be done on a computer.
        Some tasks are easy, other tasks are hard.
        (Some examples of tasks could be typing, or coding, or answering multiple-choice questions.)
        You also have a set of users $U$ who have different abilities on these tasks.
        The goal of this challenge is to produce a pseudo-code heuristic to solve each step of the challenge, supported by explanations, assumptions, or working code.
    %TODO rewrite abstract

    \end{abstract}

    \chapter{Estimate the difficulty of each task} \label{ch:q1_difficulty}

\section{Question 1} \label{sec:q1}

You want to estimate the difficulty of each task.
Each user is given a random sample of tasks from $T$ to attempt.
How can you use the performance of users (on the random task sample they are given) to estimate the difficulty of all the tasks in $T$?
(You can only use observations on the computational tasks - no biometrics, direct observation, etc.
Imagine you can see what is on a user's computer screen but nothing else.)

\section{Solution} \label{sec:q1_solution}

\subsection{Absolute and relative difficulty of a task $T_k$} \label{subsec:q1_abs_rel_difficulty}

Since it is given that users in $U$ have different abilities on tasks in $T$, when discussing the assessment of the difficulty of a task $T_k$ there are two distinct definitions of difficulty that have to be considered:

\begin{enumerate}
    \item absolute measure of difficulty for task $T_k$

    One of the ways to define the absolute measure of difficulty for a given task $T_k$, or $D_{abs}(T_k)$, is through empirically observed percentage of correct responses from all the answers, $p(T_k)$, that were provided by the subset of users $U_{T_k}$ who attempted the given task $T_k$.
    This measure indicates the difficulty of a given task as expressed by the ratio of users who were able to solve it correctly.

    \item relative measure of difficulty for task $T_k$

    Relative measure of difficulty for the task $T_k$ in relation with the user $U_i$, or $D_{rel}(T_k, U_i)$, can be defined as $D_{abs}(T_k)$ corrected for the current level of abilities of a given user $U_i$.
    This measure reflects the fact that the two users $U_i$ and $U_j$ might find the same task $T_k$ to have different level of difficulty, depending on their current abilities in a given subject.
    Furthermore, the same user $U_i$ can find the same task $T_k$ to have higher or lower level of difficulty at different points in time (e.g., if the user $U_i$ improves their skills through practice).
    $D_{rel}(T_k, U_i)$ is intended to reflect both of these differences when assessing the relative difficulty of a given task $T_k$ in relation to a given user $U_i$.
\end{enumerate}

This section is concerned with the absolute difficulty, $D_{abs}(T_k)$, of a task as reflected by the performance of the subset of users $U_{T_k}$ who have attempted the task $T_k$.
Relative task difficulty $D_{rel}(T_k, U_i)$ will be discussed further in section~\ref{subsec:q3_drel}.

\subsection{Determining absolute task difficulty $D_{abs}(T_k)$ using difficulty index} \label{subsec:q1_difficulty_index}

Difficulty index presents a basic approach to determining absolute task difficulty: difficulty $p$ of each task $T_k$ can be determined as the fraction of the correct responses provided by all the users who attempted it:

\begin{equation} \label{eq:dabs}
    \label{eq:task_difficulty}
    D_{abs}(T_k) = P(Ans=Corr|T_k) = \frac{\text{Correct answers}} {\text{Total answers}}
\end{equation}

This approach assumes that each task in $T$ was only shown to the subset of users $U_{T_k}$, for whom this task is relevant (e.g., tasks and users are grouped by school year).
Also, in this approach, individual skill level of users is not taken into account;
instead, absolute task difficulty $D_{abs}(T_k)$ can be interpreted as the expectation of the probability of an average user $U_i$ solving the given task $T_k$ correctly, as reflected by the subset of users $U_{T_k}$ who attempted this task:

\begin{equation} \label{eq:dabs_interpretation}
    D_{abs}(T_k) = E[P(Ans=Corr)|T_k, U_i \in U_{T_k}]
\end{equation}

\subsection{Uncertainty due to the size of subsets $U_{T_k}$} \label{subsec:q1_dabs_uncertainty}

Since in our case each task $T_k$ is only attempted by a subset of users $U_{T_k}$, when using the definition of absolute task difficulty $D_{abs}(T_k)$ presented in equation~\ref{eq:dabs_interpretation}, the assumption is that the expectation of probability of correct answer for an average user from the sample $U_{T_k}$ is close to that for an average user from the whole population $U$:

\begin{equation} \label{eq:dabs_assumption}
    E[P(Ans=Corr|T_k, U_i \in U_{T_k})] \approx E[P(Ans=Corr|T_k, U_i \in U)]
\end{equation}


Since it is said that users are given randomized subsets of tasks from $T$, given the large number of users and tasks, it can be assumed that for most tasks in $T$, each task $T_k$ is shown to a large random sample of users $U_{T_k}$ whose skill level and platform usage patterns are representative of the whole population of users in $U$;
in this case, absolute difficulty $D_{abs}(T_k)$ of the task $T_k$ can be assumed to offer an unbiased estimate of absolute task difficulty with respect to all users in $U$.

However, this assumption is more likely to hold true in cases where most subsets of users $U_{T_k}$ are sufficiently large, that is, enough users have attempted to solve most tasks in $T$.
To determine the uncertainty of this method, distribution of sample sizes $U_{T_k}$ can be compared with the size of the population $U$ for all tasks in $T$.
The effect of sample sizes on difficulty estimates can be mitigating by using the uncertainty of each estimate as a weight coefficient when combining empirical absolute difficulty $D_{abs}(T_k)$ with expert estimate $D_{expt}(T_k)$ to determine the composite absolute task difficulty $D_{comp}(T_k)$, which will be discussed in section~\ref{subsec:q2_dcomp}.
In addition, analysis of the distribution of $D_{abs}$ obtained empirically could be used to test the assumption and inform the design of a more meaningful scoring system that can be used to determine the absolute difficulty of a task.


    \chapter{Incorporate expert rating of difficulty of each task} \label{ch:q2_expert_rating}

\section{Question 2} \label{sec:q2}

Now suppose you are given a set $L$ of task lists $L_1, L_2, \dots, L_n$.
Each task list contains some tasks in estimated ascending difficulty order as measured by some integer $D$, where $D = 0$ for an extremely trivial task and $D = 50$ for the hardest task imaginable.
However, these given task difficulties are only estimates by experts - in practice each user who tries a task might find that task easier or harder than the expert thought it would be.
How would you use the given set of task lists along with the actual performance of users on each task to generate a single combined task list with better estimated difficulties for all tasks?
(Please assume that for any task $T_i$, the difficulty estimate $D_i$ can vary depending on the specific expert's evaluation of difficulty for $T_i$ - that is to say, experts will likely disagree on how difficult any given task is.
You can also assume that the same task may appear on multiple task lists.)

\section{Solution} \label{sec:q2_solution}

\subsection{Combining individual expert assessments into a single expert difficulty estimate $D_{expt}(T_k)$} \label{subsec:q2_dexp}

First, it is necessary to obtain the combined expert difficulty estimates $D_{expt}(T_k)$ for each task $T_k$ in $T$ by combining all the ratings $D_i(T_k)$ found for a given task on different task lists in $L$.
Since no specific weight is given to individual expert opinions, the combined expert difficulty estimate $D_{expt}(T_k)$ can be taken for each task $T_k$ as the mean of all estimates $D_i(T_k)$ from every task list in $L$ which contains the task $T_k$:

\begin{equation} \label{eq:d_expt}
    D_{expt}(T_k) = \left \lfloor \frac{1} {m} \sum \limits_{i=1}^m D_i(T_k) \right \rfloor
\end{equation}

where $m$ is the count of task lists in $L$ which include the task $T_k$.

In the equation~\ref{eq:d_expt}, $D_{expt}(T_k)$ can be left as a real number and be directly converted to a probability of a correct answer using the equation~\ref{eq:dexpt_to_dabs_conversion} that will be introduced in the following section.
However, the preferable way of converting the 0--50 scale used by experts to mark $D_{expt}$ to a 0--1 scale comparable with probabilities expressed by $D_{abs}$ involves grouping questions by $D_{expt}$(to be discussed in the following section), and thus works better for a discrete set of values.
For this purpose in equation~\ref{eq:d_expt}, the mean of all expert estimates is rounded down to the nearest integer.
This process introduces a pessimistic bias by rounding the mean difficulty values down, but since the task difficulty scale used by experts ranges from 0 to 50, this effect is assumed to be mild.

\subsection{Combining empirically determined absolute task difficulty $D_{abs}(T_k)$ with expert estimates $D_{expt}(T_k)$} \label{subsec:q2_dcomp}

In order to obtain the composite absolute difficulty measure $D_{comp}(T_k)$ of a task $T_k$, the two independently obtained measures of task difficulty $D_{abs}(T_k)$ and $D_{expt}(T_k)$ need to be combined into a single measure;
to facilitate this, they need to be converted to a comparable scale.
As was discussed in section~\ref{subsec:q1_difficulty_index}, $D_{abs}(T_k)$ represents the empirically obtained expected value of the probability that an average user in $U$ can solve the task $T_k$ (as observed by performance of users in the subset $U_{T_k}$).
In contrast, expert task difficulty estimate $D(T_k)$ is assigned using an integer scale from 0 to 50, with 0 corresponding to an extremely trivial task and 50 representing the hardest task imaginable.
Since $D_{expt}(T_k)$ was defined in section~\ref{subsec:q2_dexp} as the rounded down mean of all $D_i$ for a given task $T_k$, it will also fall onto an integer scale from 0 to 50.

To bring the two metrics onto a single composite absolute difficulty measure, $D_{comp}(T_k)$, the scale of $D_{expt}$ must be mapped onto the corresponding probabilities $D_{expt\_prob}$ of the correct answer by an average user in $U$, similar to the ones that were estimated empirically for $D_{abs}(T_k)$ in section~\ref{subsec:q1_difficulty_index}:

\begin{equation}
    \begin{split}
        D_{expt\_prob}(T_k | D_{expt}(T_k) = D_i) = E[P(Ans=Corr | D_{expt}(T_k) = D_i)]
    \end{split}
\end{equation}

The most simple way of converting the 0--50 difficulty scale used by experts to a 0--1 probability scale is through linear interpolation.
$D(T_k)=0$ can be assumed to correspond to difficulty $D_{abs}(T_k)=1.00$ (all users can solve the most trivial task), and $D(T_k)=50$ to correspond to $D_{abs}(T_k)=0.0$ (no user can solve the hardest task imaginable).
This linear relationship between expert difficulty estimate $D_{expt}$ and the absolute probability of the correct answer $D_{abs}=p$ can be represented as follows:

\begin{equation} \label{eq:dexpt_to_dabs_conversion}
    D_{expt\_prob}(T_k | D_{expt}(T_k) = D_i) = 1 - 0.02 \cdot D_{expt}
\end{equation}

where $D_i$ is an integer value from 0 to 50.

However, a better approach to converting expert difficulty estimates $D_{expt}(T_k)$ into probabilities $D_{expt\_prob}(T_k)$ is to calibrate expert task difficulty estimates to actual performance of users in $U$.
Doing this empirically would require mapping each discrete value of $D$ to the corresponding ratio of correct answers to all attempts for all the tasks in $T$ with the difficulty level of $D_i$;
this will allow to capture the non-linear relationship between expert difficulty estimates and performance of users in $U$.
This process could be repeated in order to provide an updated calibration of static expert difficulty estimates to evolving user population.

After both task difficulty measures $D_{abs}(T_k)$ and $D_{expt}(T_k)$ have been brought to the same scale (expected probabilities of the correct answer given by a user from $U$), the final composite task difficulty measure can be defined as the mean of the two difficulty measures:

\begin{equation}
    D_{comp}(T_k) = \frac{D_{abs}(T_k) + D_{expt\_prob}(T_k | D_{expt}(T_k) = D_i)} {2}
\end{equation}

Alternatively, if a given preference exists between the two difficulty measures, $D_{comp}$ can be defined as the weighted sum of the two.
For example, the uncertainty of the $D_{abs}(T_k)$ arising from the limited size of the sample of users $U_{T_k}$ who attempted the task $T_k$ (discussed in section~\ref{subsec:q1_dabs_uncertainty}) can be used as a weight coefficient;
while the number of users who tried solving $T_k$ remains small, more weight can be given to expert difficulty estimate;
as the number of users in $U_{T_k}$ grows, the role of expert assessment diminishes with more reliance being put on empirical data.

\subsection{Converting absolute composite task difficulty $D_{comp}(T_k)$ into a discrete random variable $D_{cat}(T_k)$} \label{subsec:q2_discrete_conversion}

The main task of the adaptive algorithm for designing custom task lists for each user session (to be discussed in the following sections) is to determine the sequence of task difficulties to be presented to a user.
Since for the purposes of the implementation of a planning agent it is more convenient to work with discrete action spaces rather than with continuous ones, question difficulty can be categorized by binning the values of $D_{comp}$ into a discrete categorical absolute difficulty $D_{cat}$ using a rough ``rule-of-thumb'' classification according to the difficulty index, as is shown in table~\ref{tab:difficulty_index}.

\begin{table}[h!]
    \centering
    \begin{tabular}{ c | c | c }
        \toprule
        \hline
        Task difficulty ($D_{cat}(T_k)$) & $D_{comp}(T_k)$ & \% correct \\
        \midrule
        \hline
        Very trivial & $p > 0.8$ & Over 80\% \\
        \hline
        Trivial & $0.6 < p \leq 0.8$ & From 60\% to 80\% \\
        \hline
        Moderate & $0.4 < p \leq 0.6$ & From 40\% to 60\% \\
        \hline
        Hard & $0.2 < p < \leq 0.4$ & From 20\% to 40\% \\
        \hline
        Very hard & $p \leq 0.2$ & Less than 20\% \\
        \hline
        \bottomrule
    \end{tabular}
    \caption{Categorizing questions by absolute difficulty using difficulty index.}
    \label{tab:difficulty_index}
\end{table}

So far, all the consideration has been given to rating absolute task difficulty, as defined by empirical user performance or expert judgement.
However as was discussed in section~\ref{subsec:q1_abs_rel_difficulty}, due to the difference in ability levels shown by users in $U$ on the tasks in $T$, there are two definitions of task difficulty that need to be considered for the purposes of custom task list design: absolute task difficulty for a given task $T_k$ and relative task difficulty for a given pair of task $T_k$ and user $U_i$.
In the following section, the absolute categorical task difficulty $D_{cat}(T_k)$ that was produced from the composite absolute task difficulty index $D_{comp}(T_k)$ will be used in combination with the user ranking by experience to produce relative task difficulty, $D_{rel}(T_k, U_i)$.

\subsection{Pseudocode} \label{subsec:q2_pseudocode}

\# combine expert estimates into a single absolute difficulty assessment

for each task $T_k$ in $T$:

\vspace{5mm}

$\quad$ Initialize empty list of ratings

$\quad$ for each task list in $L$ containing $T_k$:

$\quad\quad$ append rating $D_i(T_k)$ to the list of ratings

$\quad$ $D_{expt}(T_k) = \left \lfloor \frac{1} {m} \sum \limits_{i=1}^m D_i(T_k) \right \rfloor$, where $m$ in the number of task lists in $L$ containing task $T_k$

\vspace{5mm}

$\quad$ \# convert from $D_{expt}$ 0--50 scale to $D_{expt\_prob}$ 0--1 scale using linear relationship

$\quad$ \# Note: preferred conversion method involves calibration of $D_{expt\_prob}$ estimates using empirical data

\vspace{5mm}

$\quad$ $ D_{expt\_prob}(T_k | D_{expt}(T_k) = D_i) = 1 - 0.02 \cdot D_{expt} $

\vspace{5mm}

$\quad$ \# determine the combined absolute task difficulty

\vspace{5mm}

$\quad$ $D_{comp}(T_k) = \frac{D_{abs}(T_k) + D_{expt\_prob}(T_k | D_{expt}(T_k) = D_i)} {2}$

    \chapter{Question 3: adaptive algorithm for building custom task lists} \label{ch:q3_adaptive_algorithm}

\textit{Based on your answer for 2 (i.e\. given a single combined task list with estimated difficulties), suppose you wanted to build an adaptive algorithm to order the sequence of tasks each user is given.
You want to create a custom list of tasks for each user.
This custom list should give tasks of a suitable level of difficulty for each user, and you want the difficulty of tasks each user does to gradually increase over time.
How would you approach this?}

\section{Relative question difficulty and user ranking} \label{sec:relative_difficulty}

Users in $U$ are said to have different levels of abilities on tasks in $T$, which needs to be taken into account by the adaptive task list-building algorithm.
In order to make informed selection of tasks $Ti$ to be included in the custom task list $L_U_k$ to be shown to user $U_k$ in a given session, the algorithm needs to take into account the current level of abilities of the user $U_k$.
Since all the tasks in $T$ have previously been assigned a difficulty category based on the percentage of the correct answers given by all the users who attempted the task, a ranking system can be established based on the


    \chapter{Taking current user frustration into account} \label{ch:q4_user_frustration}

\section{Question 4} \label{sec:q4}
How would you modify your algorithm from part 3 to estimate a user's frustration level $F$ where $F = 0$ denotes no frustration and $F = 50$ denotes the highest level of frustration a user can safely experience without giving up at each point in time?

\section{Solution} \label{sec:q4_solution}

\subsection{Estimating user's frustration level $F_t$} \label{subsec:q4_frustration}

As was discussed in section~\ref{subsec:q3_adaptive_algorithm}, incorrect responses given by the user to presented tasks result in in-game penalties (e.g., missed attacks, lost battles, etc.);
these events are likely to contribute to a growing level of frustration experienced by a user during a session.
At the same time, users do not learn much by accomplishing only the tasks that are relatively easy for them to solve;
thus, showing the tasks of higher relative difficulty during a session would result in the improvement of learning outcomes.

When selecting a sequence of tasks to be shown to the user $U_i$ during a session, a trade-off needs to be established between the relative difficulties of tasks to be shown and the risk of causing the user to give too many wrong responses, reach their frustration limit and terminate the session.
Several basic policies presented in section~\ref{subsec:q3_adaptive_algorithm} (same difficulty policies $\pi_{easy}$, $\pi_{average}$, $\pi_{hard}$, or random policies $\pi_{random}$, $\pi_{random\_w}$ and $\pi_{hard\_easy}$) did not take into account the correctness of responses given by the user $U_i$ in the current session, and instead selected the sequence of task difficulties based solely of user ranking.
The deterministic adaptive policy $\pi_{hard\_easy}$ selected the difficulty of the next question based on the correctness of one or two past responses, but did not explicitly take into account the probabilistic nature of task difficulty definition.
This section introduces an attempt to incorporate the current frustration level $F_t$ experienced by the user $U_i$ during a session, as well as the probability of the user $U_i$ solving a task of a given relative difficulty, into the logic of the task list-building algorithm.

First, the current level of frustration $F_t$ experienced by the user $U_i$ during any state of the session needs to be established.
It is given that frustration $F$ needs to be marked on such scale so that 0 represents no frustration and 50 representing the highest level of frustration that the user $U_i$ can safely experience without terminating the session.
In order to incorporate user frustration level into a probabilistic framework, such as Markov decision process, it makes sense to associate the values of $F$ with the corresponding probability of the user $U_i$ terminating the session at a given time $P(quit|F)$.
Since $F=0$ is given to represent no frustration, it can be matched to 0\% chance of quitting:

\begin{equation}
    F = 0 \implies P(quit|F) = 0
\end{equation}

It is also given that $F=50$ represents the highest level of frustration a user can safely experience without giving up.
However, it is not clear how to interpret which level can be considered safe in probabilistic terms, since there is always some chance that a user can terminate a session on every time step.
For the purposes of this challenge, a level of user frustration $F=50$ was assumed to correspond to a 10\% chance of user terminating the session at a given time.

\begin{equation}
    F = 50 \implies P(quit|F) = 0.1
\end{equation}

By extending the scale linearly, a level of user frustration above $F=500$ is assumed to correspond to a 100\% chance of user terminating the session at a given time.

\begin{equation}
    F >= 500 \implies P(quit|F) = 1.0
\end{equation}

Overall, user frustration $F_t$ can be converted to the probability of the user terminating the session $P(quit)$ at a given time step $t$ as follows:

\begin{equation} \label{eq:frustration}
    P(quit|F) =
    \begin{cases}
        0.002 \cdot F & \text{if } F < 500 \\
        1.0 & \text{if } F >= 500
    \end{cases}
\end{equation}

For the purposes of this challenge, it is assumed that the current level of frustration is determined solely by the total number of incorrect answers given during the current session.
The frustration threshold (corresponds to $P(quit|U_i)=1$) for each user $U_i$ can be empirically determined as the mean number of incorrect answers per session for all previous sessions logged by $U_i$.
If the user $U_i$ has logged less than 10 sessions, the frustration threshold from all users in $U$ can be used instead.

\begin{equation} \label{eq:frustration_threshold}
    F(P(quit|F) = 1.0) = \frac{\sum \left( \text{Count of incorrect answers per session} \right) } {\text{Number of sessions}}
\end{equation}

\subsection{Formulating the task difficulty selection problem as a Markov decision process} \label{subsec:formulating_mdp}

A Markov decision process is a discrete time stochastic control process.
It provides a mathematical framework for modeling decision making in situations where outcomes are partly uncertain, and partly under the control of a decision maker.
MDP framework provides a formalization of the key elements in reinforcement learning (RL), such as value functions and expected reward.

An MDP expresses the problem of sequential decision making, where actions influence next states and the results.
In our case, on each time step $t$ the task list-building agent iteratively selects the difficulty level of the task to be shown to the user $U_i$;
selected difficulty influences the learning outcome (main reward $r_t$) and the probability of the user answering correctly, which in turn also influences the learning outcome as well as the frustration level experienced by user (probability of the user terminating the session at the time step $t+1$).

An MDP is four-tuple ($S, A, P, R$) including four key elements:

\begin{itemize}
    \item $S$ is the state space, with a finite set of states

    In the case of the task list-building agent, agents's state space is a sequence of tasks attempted by the user, with the next task difficulty being selected by the agent on each time step, until the user ultimately terminates the session.
    After being presented with each new task $T_t$, the user may react in one of the three possible ways:

    \begin{equation} \label{eq:mdp_state_space}
        s_t \to
        \begin{cases}
            \text{quits at the time step } t \\
            \text{tries } T_t \text{ and is right} \\
            \text{tries } T_t \text{ and is wrong}
        \end{cases}
    \end{equation}

    Each state of the agent is defined by:

    \begin{itemize}
        \item the number of tasks previously attempted by the user $U_i$ in the current session (current time step $t$)
        \item cumulative reward $G_t = \sum \limits_{i=0}^t r_i$, where $r_i$ is the reward (learning outcome) collected at every previous time step
        \item current frustration level $Fc_t$ experienced by $U_i$ (probability of the user terminating the session at the time step $t$)
    \end{itemize}

    \item $A$ is the action space, with a finite set of actions

    on each time step $t$, the agent has three possible pools from which to draw a task for the user $U_i$: easy, average, or challenging, as defined by the relative task difficulty $D_{rel}(T_k, U_i)$

    \begin{equation}
        A = \left\{ \text{easy}, \text{average}, \text{challenging} \right\}
    \end{equation}

    \item $P$ is a transition function, which defines the probability of reaching a state $s'$ from $s$ after performing an action $a$.

    \begin{table}[h!]
        \centering
        \begin{tabular}{ c || c | c | c  }
            \toprule
            \hline
            Action & \multicolumn{3}{c}{Probability of outcome} \\
            $a_t$ & quits & tries and is wrong & tries and is right \\
            \hline
            \midrule
            Easy & $Fc_t$ & $ 0.25 \cdot (1 - Fc_t) $ & $0.75 \cdot (1 - Fc_t)$\\
            \hline
            Average & $Fc_t$ & $ 0.5 \cdot (1 - Fc_t) $ & $0.5 \cdot (1 - Fc_t)$\\
            \hline
            Challenging & $Fc_t$ & $ 0.75 \cdot (1 - Fc_t) $ & $0.25 \cdot (1 - Fc_t)$\\
            \bottomrule
        \end{tabular}
        \caption{Transition matrix for the task list-building agent.
        $Fc_t$ represents current frustration level experienced by the user $U_i$ (probability of the user terminating the session at the time step $t$).
        Probabilities of the user answering a given task right or wrong are taken according to the relative task difficulty $D_{rel}(T_k, U_i)$ introduced in section~\ref{subsec:q3_drel}.}
        \label{tab:mdp_transition_matrix}
    \end{table}

    The transition function is equal to the conditional probability of reaching a target state $s'$ after performing action $a$ from state $s$.

    \begin{equation} \label{eq:mdp_transition_function)}
    P(s', s, a) = p(s'|s, a)
    \end{equation}

    As was discussed above, there are three possible actions $a$ to be taken from each state $s$ (show easy, average or challenging question next) and three possible outcomes $s'$ that may follow (user quits, user tries and is wrong, user tries and is right).
    Probability of the user quiting at a given time step $t$ is taken as $Fc_t$ (introduced in section~\ref{subsec:q4_frustration});
    therefore probability of the user attempting the task can be found as $(1 - Fc_t)$ using the rule of compliment.
    Probability of the user solving the given task can be determined using the relative task difficulty $D_{rel}(T_k, U_i)$ which was introduced in section~\ref{subsec:q3_drel} (assumed to be 75\% of right answers for relatively easy, 50\% for average, 25\% for challenging).

    Transition matrix for the task list-building agent can be defined using the table~\ref{tab:mdp_transition_matrix}.

    \item $R$ is the reward function, which determines the value received for transitioning from state $s$ to state $s'$ after performing the action $a$

    Agent's rewards $r_t$ are determined by the difficulty of the questions selected and by user's ability to answer them correctly (learning outcome).
    No negative reward is assumed by the user terminating the session;
    in this case, simply no further gain can be accomplished.

    \item Markov property

    By definition, the transition function and the reward function posses the Markov property: their values are determined only by the current state, and not from the sequence of the previous states visited.
    Markov property means that the process is memory-less and the future state depends only on the current one, and not on its history.
    This way the current state holds all the information.
    A system with such a property is called fully observable.

    In the current formulation of the task list-building agent, probability of the user quiting is assumed to be determined solely by the current frustration level at time $t$, probability to answer the questions correctly is determined solely by relative question difficulty, and the reward is determined only by the difficulty of the question and whether the question was answered correctly.
    Thus, the task list-building agent operates in a fully observable environment, where the current state holds all the information.

    \item Policy $\pi$

    The final objective of an MDP is to find a policy, $\pi$, that maximizes the cumulative reward, $ \sum \limits_{t=0}^\infty R_\pi (s_t, s_{t+1} ) $,
    where $R_\pi$ is the reward obtained at each step by following the policy, $\pi$.
    A solution of an MDP is found when a policy takes the best possible action in each state of the MDP.
    This policy is known as the optimal policy.

    \item Return

    When running a policy in an MDP, the sequence of state and action ( $ S_0, A_0, S_1, A_1, \dots $ ) is called trajectory or rollout, and is denoted by $\tau$.
    In each of agents's trajectories, a finite sequence of rewards will be collected as a result of agent's actions.
    A function of these rewards is called return and in its most simplified version, it is defined as follows:


    \begin{equation} \label{eq:mdp_return}
        G(\tau) = r_0 + r_1 + \dots + r_n = \sum \limits_{t=0}^n r_t
    \end{equation}

    A trivial but useful decomposition of return is defining it recursively in terms of return at time step $t+1$:


    \begin{equation} \label{eq:mdp_return_rec}
        G_t = r_t + G_{t+1}
    \end{equation}

    The goal of RL is to find an optimal policy, $\pi$, that maximizes the expected return as

    \begin{equation} \label{eq:mdp_rl_goal}
        argmax_\pi E_\pi[G(\tau)]
    \end{equation}

    \item Value function

    The return $G(\tau)$ provides a good insight into the trajectory's value, but doesn't give any indication of the quality of the single states visited, which can be used by the policy to choose the next best action.
    The policy has to just choose the action that will result in the next state with the highest quality.
    The value function does exactly this: it estimates the quality in terms of the expected return from a state following a policy.
    Formally, the value function is defined as follows:

    \begin{equation} \label{eq:mdp_value_function}
        V_\pi(s) = E_\pi[G|s_0=s] = E_\pi[ \sum \limits_{t=0}^k r_t |s_0 =s]
    \end{equation}

    The action-value function, similar to the value function, is the expected return from a state but is also conditioned on the first action.
    It is defined as follows:

    \begin{equation} \label{eq:mdp_action_value_function}
        Q_\pi(s, a) = E_\pi[G|s_0=s, a_0=a] = E_\pi [ \sum \limits_{t=0}^k r_t | s_0=s, a_0=a ]
    \end{equation}

    The value function and action-value function are also called the V-function and Q-function respectively, and are strictly correlated with each other since the value function can also be defined in terms of the action-value function:

    \begin{equation} \label{eq:mdp_value_action_value}
        V_\pi(s) = E_\pi[Q_\pi(s, a)]
    \end{equation}

    Knowing the optimal $Q^*$, the optimal value function is as follows:

    \begin{equation} \label{eq:mdp_optimal_value}
        V^*(s) = max_a Q^* (s, a)
    \end{equation}

    because the optimal action is $ a^* (s) = argmax_a Q^* (s, a) $.

\end{itemize}

\subsection{Modifying the task list-building algorithm to account for frustration} \label{subsec:q4_frustration_algorithm}

$V$ and $Q$ functions defined above can be estimated by running trajectories that follow the policy, $\pi$, and then averaging the values obtained.
This technique is effective and is used in many contexts, but can be very expensive considering that the return requires the rewards from the full trajectory.
The Bellman equation defines the action-value function and the value function recursively, enabling their estimations from subsequent states.
The Bellman equation does that by using the reward obtained in the present state and the value of its successor state.
Using the recursive formulation of the return defined in equation~\ref{eq:mdp_return_rec}:

\begin{equation} \label{eq:bellman_value}
    V_\pi(s) = E_\pi [G_t | s_0 = s]
    = E_\pi [r_t + G_{t+1} | s_0 = s]
    = E_\pi [r_t + V(s_{t+1}) | s_t = s, a_t \sim \pi(s_t) ]
\end{equation}

Similarly, Bellman equation can be adapted to the action-value function:

\begin{equation} \label{eq:bellman_action_value}
    \begin{split}
        Q_\pi(s, a) & = E_\pi [G_t | s_t = s, a_t = a] \\
        & = E_\pi [r_t + G_{t+1}|s_t=s, a_t=a] \\
        & = E_\pi [r_t + Q_\pi(s_{t+1}, a_{t+1})|s_t=s, a_t=a]
    \end{split}
\end{equation}

Dynamic programming (DP) is a general algorithmic paradigm that breaks up a problem into smaller chunks of overlapping subproblems, and then finds the solution to the original problem by combining the solutions of the subproblems.
DP can be used in reinforcement learning and is among one of the simplest approaches.
It is suited to computing optimal policies by being provided with a perfect model of the environment.
DP works with MDPs with a limited number of states and actions as it has to update the value of each state (or action-value), taking into consideration all the other possible states.
Since DP algorithms use tables to store value functions, it is called tabular learning.

DP uses bootstrapping, meaning that it improves the estimation value of a state by using the expected value of the following states.
DP applies the Bellman equations introduced above in equations~\ref{eq:bellman_value} and~\ref{eq:bellman_action_value} to estimate $V^*$ and/or $Q^*$, which can be done as follows:

\begin{equation} \label{eq:mdp_dp_optimal_value}
    V^*(s) = max_a E [r_t + V^*(s_{t+1}) | s_t = s, a_t = a]
\end{equation}

\begin{equation} \label{eq:mdp_dp_optimal_action_value}
    Q^*(s, a) = E [r_t + max_{a_{t+1}} Q^* (s_{t+1}, a_{t+1} | s_t = s, a_t = a)]
\end{equation}

Then, once the optimal value and action-value function are found, the optimal policy can be found by just taking the actions that maximize the expectation.

An iterative procedure that iteratively improves the value function sequence $V_0, \dots, V_k$ is called policy evaluation.
It uses the state value transition of the model, the expectation of the next state, and the immediate reward.
Policy evaluation procedure creates a sequence of improving value function using the Bellman equation:

\begin{equation}
    V_{k+1} (s) = E_\pi [r_t + V_k(s_{t+1}) | s_t = s]
    = \sum \limits_a \pi(s, a) \sum \limits_{s', r} p (s'|s, a)[r + V_k(s')]
\end{equation}

Once the value functions are improved, it can be used to find a better policy, which is called policy improvement;
it can be done as follows:

\begin{equation}
    \pi' = argmax_a Q_\pi(s, a) = argmax_a \sum \limits_{s', r} p(s'|s, a)[r + V_\pi (s')]
\end{equation}

It creates a policy, $\pi'$, from the value function, $V_\pi$, of the original policy, $\pi$.
The combination of policy evaluation and policy improvement gives rise to two algorithms to compute the optimal policy.
One is called policy iteration and the other is called value iteration.
Both use policy evaluation to monotonically improve the value function and policy improvement to estimate the new policy.
The only difference is that policy iteration executes the two phases cyclically, while value iteration combines them in a single update.

\subsection{Psudocode} \label{subsec:q4_pseudocode}

The pseudocode for the policy iteration algorithm introduced above is as follows:

$\quad$Initialize $V_\pi(s)$ and $pi(s)$ for every state $s$

\vspace{5mm}

$\quad$while $\pi$ is not stable:

\vspace{5mm}

$\quad\quad$\# policy evaluation

$\quad\quad$while $V_\pi$ is not stable:

$\quad\quad\quad$for each state $s$:

$\quad\quad\quad\quad$ $ V_\pi (s) = \sum \limits_{s', r} p (s'|s, \pi(a))[r + V_\pi(s')] $

\vspace{5mm}

$\quad\quad$\# policy improvement

$\quad\quad$for each state $s$:

$\quad\quad\quad$ $\pi = argmax_a \sum \limits_{s', r} p(s'|s, a) [r + V_\pi(s')]$

    \medskip
    \bibliography{}
    \bibliographystyle{ieeetr}

\end{document}