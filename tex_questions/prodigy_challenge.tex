%! Author = Stepan Oskin
%! Date = 14-Dec-19

% Preamble
\documentclass[11pt]{report}

% Packages
\usepackage{amsmath, subcaption}
\usepackage{graphicx}
\usepackage{titlesec, blindtext, color}

% New commands
\newcommand{\norm}[1]{\left\lVert#1\right\rVert}
\newcommand{\vect}[1]{\boldsymbol{#1}}
\newcommand{\pa}[1]{\partial{#1}}

\titleformat{\chapter}[hang]{\Huge\bfseries}{Question \thechapter\hsp\textcolor{gray75}{|}\hsp}{0pt}{\Huge\bfseries}

% Document
\begin{document}

    \title{Prodigy Algorithmic Design Challenge}

    \author{Stepan Oskin}

    \maketitle

    \begin{abstract}

        You are given a list of tasks $T = T_1, T_2, \dots, T_n$.
        Each task is to be done on a computer.
        Some tasks are easy, other tasks are hard.
        (Some examples of tasks could be typing, or coding, or answering multiple-choice questions.)
        You also have a set of users $U$ who have different abilities on these tasks.
        The goal of this challenge is to produce a pseudo-code heuristic to solve each step of the challenge, supported by explanations, assumptions, or working code.
    %TODO rewrite abstract

    \end{abstract}

    \chapter{Estimate the difficulty of each task} \label{ch:q1_difficulty}

\section{Question 1} \label{sec:q1}

You want to estimate the difficulty of each task.
Each user is given a random sample of tasks from $T$ to attempt.
How can you use the performance of users (on the random task sample they are given) to estimate the difficulty of all the tasks in $T$?
(You can only use observations on the computational tasks - no biometrics, direct observation, etc.
Imagine you can see what is on a user's computer screen but nothing else.)

\section{Solution} \label{sec:q1_solution}

\subsection{Absolute and relative difficulty of a task $T_k$} \label{subsec:q1_abs_rel_difficulty}

Since it is given that users in $U$ have different abilities on tasks in $T$, when discussing the assessment of the difficulty of a task $T_k$ there are two distinct definitions of difficulty that have to be considered:

\begin{enumerate}
    \item absolute measure of difficulty for task $T_k$

    One of the ways to define the absolute measure of difficulty for a given task $T_k$, or $D_{abs}(T_k)$, is through empirically observed percentage of correct responses from all the answers, $p(T_k)$, that were provided by the subset of users $U_{T_k}$ who attempted the given task $T_k$.
    This measure indicates the difficulty of a given task as expressed by the ratio of users who were able to solve it correctly.

    \item relative measure of difficulty for task $T_k$

    Relative measure of difficulty for the task $T_k$ in relation with the user $U_i$, or $D_{rel}(T_k, U_i)$, can be defined as $D_{abs}(T_k)$ corrected for the current level of abilities of a given user $U_i$.
    This measure reflects the fact that the two users $U_i$ and $U_j$ might find the same task $T_k$ to have different level of difficulty, depending on their current abilities in a given subject.
    Furthermore, the same user $U_i$ can find the same task $T_k$ to have higher or lower level of difficulty at different points in time (e.g., if the user $U_i$ improves their skills through practice).
    $D_{rel}(T_k, U_i)$ is intended to reflect both of these differences when assessing the relative difficulty of a given task $T_k$ in relation to a given user $U_i$.
\end{enumerate}

This section is concerned with the absolute difficulty, $D_{abs}(T_k)$, of a task as reflected by the performance of the subset of users $U_{T_k}$ who have attempted the task $T_k$.
Relative task difficulty $D_{rel}(T_k, U_i)$ will be discussed further in section~\ref{subsec:q3_drel}.

\subsection{Determining absolute task difficulty $D_{abs}(T_k)$ using difficulty index} \label{subsec:q1_difficulty_index}

Difficulty index presents a basic approach to determining absolute task difficulty: difficulty $p$ of each task $T_k$ can be determined as the fraction of the correct responses provided by all the users who attempted it:

\begin{equation} \label{eq:dabs}
    \label{eq:task_difficulty}
    D_{abs}(T_k) = P(Ans=Corr|T_k) = \frac{\text{Correct answers}} {\text{Total answers}}
\end{equation}

This approach assumes that each task in $T$ was only shown to the subset of users $U_{T_k}$, for whom this task is relevant (e.g., tasks and users are grouped by school year).
Also, in this approach, individual skill level of users is not taken into account;
instead, absolute task difficulty $D_{abs}(T_k)$ can be interpreted as the expectation of the probability of an average user $U_i$ solving the given task $T_k$ correctly, as reflected by the subset of users $U_{T_k}$ who attempted this task:

\begin{equation} \label{eq:dabs_interpretation}
    D_{abs}(T_k) = E[P(Ans=Corr)|T_k, U_i \in U_{T_k}]
\end{equation}

\subsection{Uncertainty due to the size of subsets $U_{T_k}$} \label{subsec:q1_dabs_uncertainty}

Since in our case each task $T_k$ is only attempted by a subset of users $U_{T_k}$, when using the definition of absolute task difficulty $D_{abs}(T_k)$ presented in equation~\ref{eq:dabs_interpretation}, the assumption is that the expectation of probability of correct answer for an average user from the sample $U_{T_k}$ is close to that for an average user from the whole population $U$:

\begin{equation} \label{eq:dabs_assumption}
    E[P(Ans=Corr|T_k, U_i \in U_{T_k})] \approx E[P(Ans=Corr|T_k, U_i \in U)]
\end{equation}


Since it is said that users are given randomized subsets of tasks from $T$, given the large number of users and tasks, it can be assumed that for most tasks in $T$, each task $T_k$ is shown to a large random sample of users $U_{T_k}$ whose skill level and platform usage patterns are representative of the whole population of users in $U$;
in this case, absolute difficulty $D_{abs}(T_k)$ of the task $T_k$ can be assumed to offer an unbiased estimate of absolute task difficulty with respect to all users in $U$.

However, this assumption is more likely to hold true in cases where most subsets of users $U_{T_k}$ are sufficiently large, that is, enough users have attempted to solve most tasks in $T$.
To determine the uncertainty of this method, distribution of sample sizes $U_{T_k}$ can be compared with the size of the population $U$ for all tasks in $T$.
The effect of sample sizes on difficulty estimates can be mitigating by using the uncertainty of each estimate as a weight coefficient when combining empirical absolute difficulty $D_{abs}(T_k)$ with expert estimate $D_{expt}(T_k)$ to determine the composite absolute task difficulty $D_{comp}(T_k)$, which will be discussed in section~\ref{subsec:q2_dcomp}.
In addition, analysis of the distribution of $D_{abs}$ obtained empirically could be used to test the assumption and inform the design of a more meaningful scoring system that can be used to determine the absolute difficulty of a task.


    \chapter{Incorporate expert rating of difficulty of each task} \label{ch:q2_expert_rating}

\section{Question 2} \label{sec:q2}

Now suppose you are given a set $L$ of task lists $L_1, L_2, \dots, L_n$.
Each task list contains some tasks in estimated ascending difficulty order as measured by some integer $D$, where $D = 0$ for an extremely trivial task and $D = 50$ for the hardest task imaginable.
However, these given task difficulties are only estimates by experts - in practice each user who tries a task might find that task easier or harder than the expert thought it would be.
How would you use the given set of task lists along with the actual performance of users on each task to generate a single combined task list with better estimated difficulties for all tasks?
(Please assume that for any task $T_i$, the difficulty estimate $D_i$ can vary depending on the specific expert's evaluation of difficulty for $T_i$ - that is to say, experts will likely disagree on how difficult any given task is.
You can also assume that the same task may appear on multiple task lists.)

\section{Solution} \label{sec:q2_solution}

\subsection{Combining individual expert assessments into a single expert difficulty estimate $D_{expt}(T_k)$} \label{subsec:q2_dexp}

First, it is necessary to obtain the combined expert difficulty estimates $D_{expt}(T_k)$ for each task $T_k$ in $T$ by combining all the ratings $D_i(T_k)$ found for a given task on different task lists in $L$.
Since no specific weight is given to individual expert opinions, the combined expert difficulty estimate $D_{expt}(T_k)$ can be taken for each task $T_k$ as the mean of all estimates $D_i(T_k)$ from every task list in $L$ which contains the task $T_k$:

\begin{equation} \label{eq:d_expt}
    D_{expt}(T_k) = \left \lfloor \frac{1} {m} \sum \limits_{i=1}^m D_i(T_k) \right \rfloor
\end{equation}

where $m$ is the count of task lists in $L$ which include the task $T_k$.

In the equation~\ref{eq:d_expt}, $D_{expt}(T_k)$ can be left as a real number and be directly converted to a probability of a correct answer using the equation~\ref{eq:dexpt_to_dabs_conversion} that will be introduced in the following section.
However, the preferable way of converting the 0--50 scale used by experts to mark $D_{expt}$ to a 0--1 scale comparable with probabilities expressed by $D_{abs}$ involves grouping questions by $D_{expt}$(to be discussed in the following section), and thus works better for a discrete set of values.
For this purpose in equation~\ref{eq:d_expt}, the mean of all expert estimates is rounded down to the nearest integer.
This process introduces a pessimistic bias by rounding the mean difficulty values down, but since the task difficulty scale used by experts ranges from 0 to 50, this effect is assumed to be mild.

\subsection{Combining empirically determined absolute task difficulty $D_{abs}(T_k)$ with expert estimates $D_{expt}(T_k)$} \label{subsec:q2_dcomp}

In order to obtain the composite absolute difficulty measure $D_{comp}(T_k)$ of a task $T_k$, the two independently obtained measures of task difficulty $D_{abs}(T_k)$ and $D_{expt}(T_k)$ need to be combined into a single measure;
to facilitate this, they need to be converted to a comparable scale.
As was discussed in section~\ref{subsec:q1_difficulty_index}, $D_{abs}(T_k)$ represents the empirically obtained expected value of the probability that an average user in $U$ can solve the task $T_k$ (as observed by performance of users in the subset $U_{T_k}$).
In contrast, expert task difficulty estimate $D(T_k)$ is assigned using an integer scale from 0 to 50, with 0 corresponding to an extremely trivial task and 50 representing the hardest task imaginable.
Since $D_{expt}(T_k)$ was defined in section~\ref{subsec:q2_dexp} as the rounded down mean of all $D_i$ for a given task $T_k$, it will also fall onto an integer scale from 0 to 50.

To bring the two metrics onto a single composite absolute difficulty measure, $D_{comp}(T_k)$, the scale of $D_{expt}$ must be mapped onto the corresponding probabilities $D_{expt\_prob}$ of the correct answer by an average user in $U$, similar to the ones that were estimated empirically for $D_{abs}(T_k)$ in section~\ref{subsec:q1_difficulty_index}:

\begin{equation}
    \begin{split}
        D_{expt\_prob}(T_k | D_{expt}(T_k) = D_i) = E[P(Ans=Corr | D_{expt}(T_k) = D_i)]
    \end{split}
\end{equation}

The most simple way of converting the 0--50 difficulty scale used by experts to a 0--1 probability scale is through linear interpolation.
$D(T_k)=0$ can be assumed to correspond to difficulty $D_{abs}(T_k)=1.00$ (all users can solve the most trivial task), and $D(T_k)=50$ to correspond to $D_{abs}(T_k)=0.0$ (no user can solve the hardest task imaginable).
This linear relationship between expert difficulty estimate $D_{expt}$ and the absolute probability of the correct answer $D_{abs}=p$ can be represented as follows:

\begin{equation} \label{eq:dexpt_to_dabs_conversion}
    D_{expt\_prob}(T_k | D_{expt}(T_k) = D_i) = 1 - 0.02 \cdot D_{expt}
\end{equation}

where $D_i$ is an integer value from 0 to 50.

However, a better approach to converting expert difficulty estimates $D_{expt}(T_k)$ into probabilities $D_{expt\_prob}(T_k)$ is to calibrate expert task difficulty estimates to actual performance of users in $U$.
Doing this empirically would require mapping each discrete value of $D$ to the corresponding ratio of correct answers to all attempts for all the tasks in $T$ with the difficulty level of $D_i$;
this will allow to capture the non-linear relationship between expert difficulty estimates and performance of users in $U$.
This process could be repeated in order to provide an updated calibration of static expert difficulty estimates to evolving user population.

After both task difficulty measures $D_{abs}(T_k)$ and $D_{expt}(T_k)$ have been brought to the same scale (expected probabilities of the correct answer given by a user from $U$), the final composite task difficulty measure can be defined as the mean of the two difficulty measures:

\begin{equation}
    D_{comp}(T_k) = \frac{D_{abs}(T_k) + D_{expt\_prob}(T_k | D_{expt}(T_k) = D_i)} {2}
\end{equation}

Alternatively, if a given preference exists between the two difficulty measures, $D_{comp}$ can be defined as the weighted sum of the two.
For example, the uncertainty of the $D_{abs}(T_k)$ arising from the limited size of the sample of users $U_{T_k}$ who attempted the task $T_k$ (discussed in section~\ref{subsec:q1_dabs_uncertainty}) can be used as a weight coefficient;
while the number of users who tried solving $T_k$ remains small, more weight can be given to expert difficulty estimate;
as the number of users in $U_{T_k}$ grows, the role of expert assessment diminishes with more reliance being put on empirical data.

\subsection{Converting absolute composite task difficulty $D_{comp}(T_k)$ into a discrete random variable $D_{cat}(T_k)$} \label{subsec:q2_discrete_conversion}

The main task of the adaptive algorithm for designing custom task lists for each user session (to be discussed in the following sections) is to determine the sequence of task difficulties to be presented to a user.
Since for the purposes of the implementation of a planning agent it is more convenient to work with discrete action spaces rather than with continuous ones, question difficulty can be categorized by binning the values of $D_{comp}$ into a discrete categorical absolute difficulty $D_{cat}$ using a rough ``rule-of-thumb'' classification according to the difficulty index, as is shown in table~\ref{tab:difficulty_index}.

\begin{table}[h!]
    \centering
    \begin{tabular}{ c | c | c }
        \toprule
        \hline
        Task difficulty ($D_{cat}(T_k)$) & $D_{comp}(T_k)$ & \% correct \\
        \midrule
        \hline
        Very trivial & $p > 0.8$ & Over 80\% \\
        \hline
        Trivial & $0.6 < p \leq 0.8$ & From 60\% to 80\% \\
        \hline
        Moderate & $0.4 < p \leq 0.6$ & From 40\% to 60\% \\
        \hline
        Hard & $0.2 < p < \leq 0.4$ & From 20\% to 40\% \\
        \hline
        Very hard & $p \leq 0.2$ & Less than 20\% \\
        \hline
        \bottomrule
    \end{tabular}
    \caption{Categorizing questions by absolute difficulty using difficulty index.}
    \label{tab:difficulty_index}
\end{table}

So far, all the consideration has been given to rating absolute task difficulty, as defined by empirical user performance or expert judgement.
However as was discussed in section~\ref{subsec:q1_abs_rel_difficulty}, due to the difference in ability levels shown by users in $U$ on the tasks in $T$, there are two definitions of task difficulty that need to be considered for the purposes of custom task list design: absolute task difficulty for a given task $T_k$ and relative task difficulty for a given pair of task $T_k$ and user $U_i$.
In the following section, the absolute categorical task difficulty $D_{cat}(T_k)$ that was produced from the composite absolute task difficulty index $D_{comp}(T_k)$ will be used in combination with the user ranking by experience to produce relative task difficulty, $D_{rel}(T_k, U_i)$.

\subsection{Pseudocode} \label{subsec:q2_pseudocode}

\# combine expert estimates into a single absolute difficulty assessment

for each task $T_k$ in $T$:

\vspace{5mm}

$\quad$ Initialize empty list of ratings

$\quad$ for each task list in $L$ containing $T_k$:

$\quad\quad$ append rating $D_i(T_k)$ to the list of ratings

$\quad$ $D_{expt}(T_k) = \left \lfloor \frac{1} {m} \sum \limits_{i=1}^m D_i(T_k) \right \rfloor$, where $m$ in the number of task lists in $L$ containing task $T_k$

\vspace{5mm}

$\quad$ \# convert from $D_{expt}$ 0--50 scale to $D_{expt\_prob}$ 0--1 scale using linear relationship

$\quad$ \# Note: preferred conversion method involves calibration of $D_{expt\_prob}$ estimates using empirical data

\vspace{5mm}

$\quad$ $ D_{expt\_prob}(T_k | D_{expt}(T_k) = D_i) = 1 - 0.02 \cdot D_{expt} $

\vspace{5mm}

$\quad$ \# determine the combined absolute task difficulty

\vspace{5mm}

$\quad$ $D_{comp}(T_k) = \frac{D_{abs}(T_k) + D_{expt\_prob}(T_k | D_{expt}(T_k) = D_i)} {2}$

    \chapter{Question 3: adaptive algorithm for building custom task lists} \label{ch:q3_adaptive_algorithm}

\textit{Based on your answer for 2 (i.e\. given a single combined task list with estimated difficulties), suppose you wanted to build an adaptive algorithm to order the sequence of tasks each user is given.
You want to create a custom list of tasks for each user.
This custom list should give tasks of a suitable level of difficulty for each user, and you want the difficulty of tasks each user does to gradually increase over time.
How would you approach this?}

\section{Relative question difficulty and user ranking} \label{sec:relative_difficulty}

Users in $U$ are said to have different levels of abilities on tasks in $T$, which needs to be taken into account by the adaptive task list-building algorithm.
In order to make informed selection of tasks $Ti$ to be included in the custom task list $L_U_k$ to be shown to user $U_k$ in a given session, the algorithm needs to take into account the current level of abilities of the user $U_k$.
Since all the tasks in $T$ have previously been assigned a difficulty category based on the percentage of the correct answers given by all the users who attempted the task, a ranking system can be established based on the



    \medskip
    \bibliography{}
    \bibliographystyle{ieeetr}

\end{document}